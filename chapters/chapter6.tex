\chapter{Discussion}
\thispagestyle{fancy}

\section{Gathered Insights}
\subsection{General Intention to Use}
The majority of users expressed the intention to use the application. The average user would recommend \emph{Allergy Scan} to friends and keep on using the application while purchasing food items, both at home and abroad (\cref{fig:overall-bi}). \emph{Allergy Scan} is perceived to be time-saving and helpful in identifying critical allergens. However, the trust (PE 3) in the application \emph{Allergy Scan} is an area that leaves room for further improvement (\cref{fig:overall-pe}). In addition, the average user is indifferent between using the app (EE 1), on the one hand, and reading the product's label, on the other (\cref{fig:overall-ee}).

\subsection{Improvement through Tailoring}
The comparison of the \emph{tailored} and \emph{non-tailored} sub-group puts forth that a user-specific tailored result has no positive impact on the user's intention to you the app. The received feedback from the \emph{non-tailored} sub-group was even slightly better (\cref{fig:feedback-tailoring}). User from this sub-group also used the app three days longer and scanned over 50\% more products on average.

The fact, that a tailored result leads seems to perform slightly weaker might be considered as surprising, since one would presume that a customized user experience might be more preferable. A possible explanation is that the omission of obviously contained allergens may cause a loss of confidence in the reliability of the app among users with a tailored result. This finding, however, hints that providing a tailored result, which is only possible with a digital solution, does not add much value to the user.

\subsection{Improvement through Visualization}
The visual presentation by means of icons generated a more positive feedback than the text-only approach. Users within the \emph{visual} sub-group rated the intention to use with an average of 4.22, while the \emph{textual} sub-group's average rating of 3.68 was significantly lower (\cref{fig:feedback-presentation}). Interestingly, the users with a textual response used the application for two days more and captured four more products on average compared to the users with a visual response (\cref{tab:usage-presentation}).

\subsection{Effects of Language Barriers}
Non-fluent German speakers tend to perceive the application as more useful than the users who speak German fluently. This, again, correlates with all three constructs (\cref{fig:feedback-german}) and the same conclusion can be drawn when taking the device language into consideration (\cref{fig:feedback-lang}). The remarkable gap in performance expectancy (4.67 to 3.55) substantiates the fact, that the current text-only approach (\cref{sec:problem}) is not sufficient for people with an international background.

\subsection{Additional Drivers and Barriers for Intention to Use}
The user's gender has a drastic impact on the usage of the app: each measurement of interaction for male users is at least twice as high compared to their female peer (\cref{tab:usage-gender}). There is also a noticeable higher feedback provided men with a gap up to 0.5 among the three constructs (\cref{fig:feedback-gender}).

There is also evidence for a correlation between the feedback values and the user's age. Younger users tend to signal a stronger intend to use the app compared to older participants (\cref{fig:feedback-age}). Considering the usage data (\cref{tab:usage-age}), younger users seem to use the app slightly longer, but less intensive.

The travelling attitude of the users shows no noticeable effect on the intention to use, although participants who spend at least two weeks a year in a country whose language they do not speak provided a slightly more positive feedback (\cref{fig:feedback-travelling}). However, the majority of the sample speaks German fluently and the app is only available in Switzerland. In order to make a reliable statement about the effect of this aspect, more feedback of non-fluent German speakers in Switzerland has to be collected. 

The amount of allergies show a big difference among all three constructs (\cref{fig:feedback-allergies}). Users with multiple food allergies rate the intention to use with an average value between 4.17 and 4.25, while the average value from users with one allergy is only 3.45. Also, users with one allergy are the only group where one construct (performance expectancy) falls below the neutral value of 3. A possible explanation for the tremendous gap in performance expectancy (4.06 to 2.87) is the greater effort of skimming the product label for multiple allergens and the greater risk of accidentally selecting a product with a critical allergen.

Users whose scans were successfully found on the Eatfit API with a ratio below 45\% showed a weaker intention to use compared to users with a higher ratio. A ratio above 45\% seems to result in a solid positive feedback around a value of 4 (\cref{fig:feedback-ratio}).

The criteria if a user checks at least once a week a food item's label for allergies does not have a noticeable effect on the intention to use in general. However, users doing weekly product label checks signal a stronger performance expectancy and effort expectancy (\cref{fig:feedback-research}) and they scanned twice as many products (\cref{tab:usage-research}).


\section{Conclusion}
Despite the unfavorable conversion rate, plenty of revealing insight was provided during this experiment. The overall feedback of the users about \emph{Allergy Scan} is very positive and thus one can say that the application has proved its right to exist. The app is perceived to be useful and it appears to support people who suffer from food allergies and/or food intolerances in their daily lives. Especially individuals with multiple food allergies and non-native German speakers appreciate this modern solution, that helps them deal with their food allergies.

However, the assertion that an application could replace the need for physical product information goes too far. Apps, such as \emph{Allergy Scan}, might rather complement physical allergen descriptions. They provide an interesting opportunity for supporting people who suffer from food allergies and do not speak the language in which the allergens are declared physically. 

The most revealing insight of this research paper is that the presentation of allergens as icons is clearly superior to a text-only approach. The greatest benefit is that the icons can be easily understood by anyone and thus the threats resulting from language barriers can be alleviated. In addition, tracking down an icon takes less time and effort and provides more security than skimming a text for specific keywords.

\section{Limitation}
The feedback and usage data gave an interesting insight into how potential allergens can be displayed, however, the sample at hand is quite small. Unfortunately, a low ratio of successful product queries had a significant impact on the conversion rate and many users ceased using the application due to that. With more user feedback, the collected data could be split into further clusters and analyzed more explicitly. It should be noted that the Eatfit API is continuously extending its database with the addition of further products to provide a better product coverage. The application \emph{Allergy Scan} will continue to be available and keep collecting data in order to better understand its users.

\section{Future Work}
This thesis provides the foundation for further studies in a promising field. Due to the finding that icons might become a replacement for the current text-only declaration of allergens, a stronger focus can be laid on the visualization of icons. Evaluating a set of icons which can be recognized and understood by every individual, irrespective of culture and origin, would remove the threats that language barriers pose and eliminate the cost and effort of translating the content of the product package.

Moreover, by achieving a greater sample through further promoting the \emph{Allergy Scan} and improving the product coverage of the Eatfit API, the gathered data can be analyzed in more detail and the users can be clustered into smaller clusters which possibly provide even more specific insights. 