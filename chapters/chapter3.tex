\chapter{Research Design}
\thispagestyle{fancy}

\section{Test Groups}
\label{sec:text-groups}
All participants are randomly allocated within four test groups to get an insight in how the result of the screening of a food item’s allergy profile is perceived by an individual user. These test groups differ in two independent dimensions: the visualization of the allergens and the tailoring of the scanning result in respect of the allergens to be displayed (\cref{sub:result}).

\subsection{Tailoring}
The users will be separated into two different sub-groups, namely the \emph{tailored} and the \emph{non-tailored} sub-group. The participants who have been allocated to the \emph{non-tailored} sub-group will receive a full allergy profile of the product after the scan, independently of their predefined allergy profile. For those participants who have been appointed to the \emph{tailored} sub-group, \emph{Allergy Scan} will only portray the allergens that are consistent with the allergens in their predefined allergy profile. 

\subsection{Visualization}
The way in which the allergens are displayed after the product scan will also vary depending on which sub-group the participant has been assigned to. One sub-group will receive a textual response, whereas the the other sub-group will be confronted with a visual response. A user within the \emph{visual} sub-group receives the product’s allergy profile as a list of icons (e.g. peanut icon). If the product does not contain any critical allergens, then a green success indicator is shown for the \emph{visual} sub-group. The \emph{textual} sub-group on the other hand receives the product's allergy profile in the form of a text. In case the product does not contain any critical allergens, a text hint will convey the information to the user. In order to get a deeper insight on the international scalability of the alternative, the textual result will not be translated to the user's preferred language, it will only be displayed in German. This aspect will be taken into consideration when analyzing the user's feedback.

Therefore, the four distinct test groups are defined as:

\begin{table}[H]
\centering
\begin{tabular}{c c | c c}

\multicolumn{2}{c}{} & \multicolumn{2}{c}{\textbf{Tailoring}}\\

\multicolumn{2}{c}{}  & Tailored & Non-Tailored \\
\hline
\multirow{2}{*}{\textbf{Visualization}}    &   Textual &   Tailored Textual  &  Non-Tailored Textual\\
                &   Visual  &   Tailored Visual  &  Non-Tailored Visual\\

\bottomrule
\end{tabular}
\caption{Test groups}
\end{table}

\section{Data Gathering}
To be able to analyze how an individual interacts with the application the data will be collected and stored anonymously. An individual’s data can be categorized into two major sets:

\subsection{Profiling}
\label{sub:profiling}
To be able to draw any conclusion about an individual, he or she must be known and understood. Therefore, the user’s demographic and behavioral metrics will be analyzed in this study.

\begin{table}[H]
\centering
\begin{tabular}{l l}
\toprule
Metric & Values\\
\midrule
Allergy profile   &	14 types of allergens [REF to table in Motivation] \\
Preferred language   &	German or English \\
Language skills in German  &	Native, fluent, basics or none \\
Frequently traveling   &	The participant spends at least two weeks a year in a country whose language he or she does not speak. \\
\bottomrule
\end{tabular}
\caption{Profiling}
\end{table}

\subsection{Usage Behaviour}
\label{sub:behaviour}
A further source of valuable user data is how the individual interacts with the application itself. For this purpose, specific events must be tracked in order to be able to analyze and assess the intensity of the user's interaction with the application.
\begin{table}[H]
\centering
\begin{tabular}{p{4cm} p{10cm}}
\toprule
Metric & Explanation\\
\midrule
\textbf{Amount of scans}   &	Amount of products the participant has scanned \\
%% \textbf{Success Ratio}   &	Ratio of scans which resulted in a successful response \\
\textbf{Days of usage}   &	Amount of days on which the participant has scanned at least one product \\
\textbf{Product captures}   &	Amount of products the participant has captured as they were not found. \\
%% \textbf{Product Reports} &	Amount of reports the participant has posted, as there might be an error with the product response. \\
\bottomrule
\end{tabular}
\caption{Usage behavior}
\end{table}
\label{table:user-behavior}

\section{Survey Structure}
\label{sec:survey}
After ensuring that the participants have engaged with the application thoroughly, they will be encouraged to take a survey. The results of this survey will be used to examine the research questions (\cref{sec:research-questions}) and thus represent the foundation of this thesis. The survey is built on the three constructs described in \cref{sec:utaut} and each question or statement is answered with an option generating a value from one (strongly agree) to five (strongly disagree). The exact wording and answer-scheme in German and English can be found in \cref{chap:feedback-schema}.

\begin{table}[H]
\centering
\begin{tabular}{l l}

\toprule
Label & Question\\
\midrule
BI 1    &   How would rate “Allergy Scan”? \\
BI 2	&   Will you keep using "Allergy Scan" in the future for regular purchases? \\
BI 3	&   Will you keep using "Allergy Scan" in the future while travelling abroad? \\
BI 4	&   Would you recommend "Allergy Scan" to your friends? \\
\bottomrule

\end{tabular}
\caption{Behavioral Intention}
\end{table}

\begin{table}[H]
\centering
\begin{tabular}{p{1cm} p{14cm}}

\toprule
Label & Sub-construct \\
\midrule
EE 1    &	Perceived ease of use \\
EE 2    &	Complexity \\
EE 3    &	Ease of use \\

\bottomrule

\end{tabular}
\caption{Effort expectancy}
\end{table}

\begin{table}[H]
\centering
\begin{tabular}{l l}

\toprule
Label & Question \\
\midrule
PE 1    &	"Allergy Scan" supports me to identify products that I am allergic to.  \\
PE 2    &	"Allergy Scan" saves me time since I do not need to read the ingredient list to check for potential allergens. \\
PE 3    &	After scanning a product with "Allergy Scan", do you double-check the label? \\

\bottomrule

\end{tabular}
\caption{Performance Expectancy}
\end{table}

\section{Research Questions}
\label{sec:research-questions}
\subsection{Overall Intention to Use}
\textit{\enquote{Does a \textbf{majority of users} state the \textbf{intention to use} Allergy Scan?}}\\

Based on the \emph{intention to use} section of the survey (\cref{tab:intention-to-user}), the general feedback among all users is evaluated, to come to know if most users signal an intention to use the application. The present research paper considers how the user rates the app in general, if the user intends to keep using the app and if the user would recommend the app to other people. These analytics provide information on whether the app is a feasible solution for the underlying problem.

\subsection{Improvement through Tailoring}
\label{rc:tailoring}
\textit{\enquote{How does the integration of a \textbf{user-specific tailored result} improve the \textbf{usage behavior and intention to use} and its underlying constructs?}}\\

The consolidated feedback data provided by the \emph{tailored} sub-group is compared to the data from the \emph{non-tailored} sub-group to find out how a user-specific tailored result affects the user's intention to use. The two sub-groups are also compared in regard to the user's usage behavior, such as the number of products scanned, the number of days using the app and number of captures of missing products.

\subsection{Improvement through Visualization}
\textit{\enquote{How does the integration of \textbf{visual icons} improve the \textbf{usage behavior and intention to use} and its underlying constructs?}}\\

Similarly to \cref{rc:tailoring}, the feedback data and usage behavior of users from the \emph{textual} sub-group is consolidated and compared to the \emph{visual} sub-group, in order to find out how a user-specific tailored result affects the user's intention to use and usage behavior.

\subsection{Effects of Language Barriers}
\textit{\enquote{How does the \textbf{intention to use and usage behavior} differ between \textbf{fluent speakers and non-fluent local speakers}?}}\\

The feedback data of fluent and non-fluent local speakers is evaluated to see whether there is a significant difference in their intention to use and usage behavior. This examination takes into consideration the German language level and the language setting of the user’s device.

%% \subsection{Effects of Travel Frequency}
%% \textit{\enquote{How does the \textbf{intention to use and usage behavior} differ between \textbf{frequent and infrequent travellers}?}}\\

%% \subsection{Effects of Allergy Profile}
%% \textit{\enquote{How does the \textbf{intention to use and usage behavior} differ between users with or without a \textbf{distinct allergy profile}?}}\\


\subsection{Additional Drivers and Barriers for Intention to Use}
\textit{\enquote{Which further constructs are \textbf{drivers} and \textbf{barriers} for \textbf{intention to use} Allergy Scan?}}\\

The collected data will be analyzed through clustering the users by gender, age, travel frequency, allergy profile and ratio of successfully queried products. In doing so, one can evaluate in a next step which of the above-mentioned aspects have a positive or negative impact on the intention to use and influence the user in his or hers behavior.
