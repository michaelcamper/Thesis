\section{Test Groups}
In order to get an insight in how the result of the screening of a food item’s allergy profile is perceived by an individual user, all participating users of the application are randomly allocated within four test groups. These test groups differ in two independent dimensions: The tailoring of the scanning result in respect of the to be displayed allergens and the presentation of the displayed allergens.

\subsection{Tailoring}
All users will be separated into two different sub-groups which determine what allergens the result after scanning a product will contain. After scanning a product, the complete allergy profile of the item will be displayed to the "non-tailored" sub-group, regardless of their predefined allergy profile. On the other hand, users within the "tailored" sub-group will only be presented with the allergens intersecting their predefined allergy profile.

\subsection{Presentation}
The way how the allergens contained in the result of the product scan are displayed will also differ across two sub-groups: One of these groups receives a textual response and the other one receives a visual response. A user within the "visual" sub-group receives the product's allergy profile as a list of icons (e.g. a peanut icon) and in the case of no critical allergens are present, a green success indicator. The "textual" sub-group receives the product allergy profile in the form text or, if there are no allergens to be displayed, a text message, which states that no critical allergy were detected. In order to get a further insight on the international scalability of the alternative, the textual result will not be translated to the user's preferred language, but only be displayed in German. This aspect will be taken into consideration when analyzing the user's feedback.\\
\\
Therefore, the four distinct test groups are defined as:

\begin{table}[H]
\centering
\begin{tabular}{c c | c c}

\multicolumn{2}{c}{} & \multicolumn{2}{c}{\textbf{Tailoring}}\\

\multicolumn{2}{c}{}  & Tailored & Non-Tailored \\
\hline
\multirow{2}{*}{\textbf{Visualization}}    &   Textual &   Tailored Textual  &  Non-Tailored Textual\\
                &   Visual  &   Tailored Visual  &  Non-Tailored Visual\\

\bottomrule
\end{tabular}
\caption{Test groups}
\end{table}

\section{Data Gathering}
In order to be able to analyze how an individual interacts with the application, data will be gathered. All user data is being collected and stored anonymously. The data collected from an individual can be categorized into two major sets:

\subsection{Profiling}
To be able to make any conclusion about an individual, the individual must be known and understood. Therefor, there is plenty demographic and behavioural metrics about the user collected.
\begin{table}[H]
\centering
\begin{tabular}{l l}
\toprule
Metric & Values\\
\midrule
Allergy profile   &	14 types of allergens [REF to table in Motivation] \\
Preferred language   &	German or English \\
Language skills in German  &	Native, fluent, basics or none \\
Frequently traveling   &	The participant spends at least two weeks a year in a country whose language he or she does not speak. \\
\bottomrule
\end{tabular}
\caption{Profiling}
\end{table}

As this paper is part of a more comprehensive research on nutrition, further data about the participants are being collected, but are not taken into consideration for evaluating the results of this study. \\
These additional metrics are: Gender, age, weight, height, highest level of education, income, physical activity, type of diet, food avoidance and active nutrition research.

\subsection{Usage Behaviour}
Another valuable aspect gathering user data is how the individual interacts with the application itself. For this purpose, specific events must be tracked in order to be able to analyze and assess the intensity of the user's interaction with the application.
\begin{table}[H]
\centering
\begin{tabular}{l l}
\toprule
Metric & Explanation\\
\midrule
Scans   &	Amount of products the participant has scanned \\
Success Ratio   &	Ratio of scans which resulted in a successful response \\
Days of Usage   &	Amount of days on which the participant has scanned at least one product \\
Products Captured   &	Amount of products the participant has captured as they were not found. \\
Product Reports &	Amount of reports the participant has posted, as there might be an error with the product response. \\
\bottomrule
\end{tabular}
\caption{Usage Behavior}
\end{table}

\section{Survey}
After ensuring that the participant has interacted with the application in a serious manner, the participant will be encouraged to take a survey. The results of this survey will be used to examine the research questions [REF] and build the foundation of this thesis. The survey is built on of three main constructs and each question or statement is answered with an option generating a value from one to five.

\subsection{Behavioral Intention}
[Explain Construct]
\begin{table}[H]
\centering
\begin{tabular}{p{1cm} p{14cm}}

\toprule
Label & Aspect\\
\midrule
BI 1    &   Overall rating of the Application \\
BI 2	&   Future intention to use for regular purchases \\
BI 3	&   Future intention to use while  travelling abroad \\
BI 4	&   Intention to recommend to friends \\
\bottomrule

\end{tabular}
\caption{Intention to use}
\label{tab:intention-to-user}
\end{table}
[Explain Questions]

\subsection{Effort Expectancy}
[Explain Construct]
\begin{table}[H]
\centering
\begin{tabular}{p{1cm} p{14cm}}

\toprule
Label & Sub-construct \\
\midrule
EE 1    &	Perceived ease of use \\
EE 2    &	Complexity \\
EE 3    &	Ease of use \\

\bottomrule

\end{tabular}
\caption{Effort expectancy}
\end{table}
[Explain Questions]

\subsection{Performance Expectancy}
[Explain Construct]
\begin{table}[H]
\centering
\begin{tabular}{p{1cm} p{14cm}}

\toprule
Label & Sub-constract \\
\midrule
PE 1    &	Job-fit \\
PE 2    &	Relative advantage (Time-saving) \\
PE 3    &	Perceived usefulness \\

\bottomrule

\end{tabular}
\caption{Performance expectancy}
\end{table}
[Explain Questions]

\section{Research Questions}
\subsection{RQ1}
\textit{\large Does a majority of users state the intention to use this novel barcode-scanning allergen feedback app?}\\



\subsection{RQ2}
\textit{\large 
Which constructs are drivers for intention to use this novel barcode-scanning allergen feedback app?}\\



\subsection{RQ3}
\textit{\large Which constructs are barriers for intention to use this novel barcode-scanning allergen feedback app?}\\



\subsection{RQ4}
\textit{\large How does the integration of user-specific tailored allergy labels improve the usage behavior and intention to use and its underlying constructs to use this novel barcode-scanning allergen feedback app?}\\


\subsection{RQ6}
\textit{\large How does the integration of visual icons improve the usage behavior and intention to use and its underlying constructs to use this novel barcode-scanning allergen feedback app?}\\




\subsection{RQ7}
\textit{\large Does a majority of users state the intention to use this novel barcode-scanning allergen feedback app?}\\



\subsection{RQ8}
\textit{\large Does a majority of users state the intention to use this novel barcode-scanning allergen feedback app?}\\


