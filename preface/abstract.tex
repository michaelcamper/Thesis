\chapter*{Abstract}

Approximately 30 percent of Switzerland’s population is affected by a form of food allergy or food intolerance. Product manufactures are legally required to explicitly label 14 potentially harmful ingredients and products derived therefrom. Current best practice is a text-only approach, which potentially creates a language barrier for people not being capable of understanding the local language.


In order to get examine a superior solution, the smartphone application \emph{Allergy Scan} was created and published. The app allows users to retrieve a product’s allergens profile by scanning its bar code. A built-in A/B-testing separates the users into four test groups which differ from each other in the way the product’s allergens are displayed and filtered.


A rather small, but qualitative sample of user feedback and usage data was acquired and analyzed. The overall feedback by the users was positive and the app is considered as useful by the majority of users. The main finding from this experiment is that a visual presentation of contained allergens by means of icons significantly affected the user’s feedback and usage behavior. However, a user-specific tailored result did not add any noticeable value to the user experience. Also, users suffering from more than one food allergy and non-fluent local speakers especially signaled a positive echo.


As visualizing allergens with the help of icons signaled a positive feedback, it offers the opportunity to examine the definition of a set of icons which can be understood by the majority of people. Further insights can also be achieved by expanding the sample of user feedback over time.
